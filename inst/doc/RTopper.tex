%%%%%%%%%%%%%%%%%%%%%%%%%%%%%%%%%%%%%%%%%%%%%%%%%%
%% Luigi Marchionni
%% August 21 2008
%%%%%%%%%%%%%%%%%%%%%%%%%%%%%%%%%%%%%%%%%%%%%%%%%%

%%%%%%%%%%%%%%%%%%%%%%%%%%%%%%%%%%%%%%%%%%%%%%%%%% 
%% Begin Document
\documentclass[11pt]{article}

%%%%%%%%%%%%%%%%%%%%%%%%%%%%%%%%%%%%%%%%%%%%%%%%%% 
%% Preamble
%%%%%%%%%%%%%%%%%%%%%%%%%%%%%%%%%%%%%%%%%%%%%%%%%
%% Luigi Marchionni
%% Preamble
%%%%%%%%%%%%%%%%%%%%%%%%%%%%%%%%%%%%%%%%%%%%%%

%% To use double spacing
\usepackage{setspace}
% \doublespacing

%% For graphics, fonts, figures
\usepackage{latexsym}
\usepackage{amsmath}
\usepackage{amssymb}
\usepackage{amsfonts}
\usepackage{amsxtra}

%% For graphics, fonts, figures
%\usepackage{graphicx, helvet, floatflt, comment, color}
\usepackage{graphicx, helvet, comment, color}
\usepackage{wrapfig, subfigure}
\usepackage{vmargin}
\usepackage{picinpar, fancyhdr} 
\usepackage{epsfig}
\usepackage{epstopdf}

%% to control floats
\usepackage{float} 

%% Set margins
%\usepackage[left=0.75 in,top=0.75 in,right=0.75 in,bottom=0.75 in]{geometry}
\usepackage{geometry} 
%\geometry{letterpaper}

%% Activate to begin paragraphs with an empty line
\usepackage[parfill]{parskip}

%% to make Celsius
\usepackage{textcomp}

%% For bibliography
\usepackage{natbib}

%% for Sweave files
%\usepackage{Sweave}

%% For URLs formatting
\usepackage{url}

%%to make hyperlink to html
\usepackage{hyperref}
%\usepackage[dvips, colorlinks=true]{hyperref}

%% To put legend on the side
%\usepackage{sidecap}

%% For long table floating across pages
\usepackage{longtable}

%% For page orientation
\usepackage{lscape}



%% Document layout
\parindent 0in
\setpapersize{USletter}
\setmarginsrb{1truein}{0.5truein}{1truein}{0.5truein}{16pt}{30pt}{0pt}{20truept}
\setlength{\emergencystretch}{2em}

\usepackage{Sweave}

%\VignetteIndexEntry{RTopper user's manual}

%% New commands for R stuff
\newcommand{\Robject}[1]{{\texttt{#1}}}
\newcommand{\Rfunction}[1]{{\texttt{#1}}}
\newcommand{\Rpackage}[1]{{\texttt{#1}}}
\newcommand{\Rclass}[1]{{\texttt{#1}}}
\newcommand{\Rmethod}[1]{{\texttt{#1}}}
\newcommand{\Rfunarg}[1]{{\texttt{#1}}}

\begin{document}

\title{The RTopper package: perform run Gene Set Enrichment across genomic platforms}

\author{Luigi Marchionni \\
Department of Oncology \\
Johns Hopkins University \\
email: \texttt{marchion@jhu.edu}}

\singlespacing
\maketitle
\tableofcontents


\section{Overview}
Gene Set Enrichment (GSE) analysis has been widely use to assist the  interpretation
of gene expression data. We propose here to apply GSE for the integration of genomic 
data obtained from distinct analytical platform.

In the present implementation of the \Rpackage{RTopper} GSE analysis is performed
using the \Rfunction{geneSetTest} function from the \Rpackage{limma} package
\cite{Smyth2005a,Smyth2004,Smyth2005}.
This function enables testing the hypothesis that a specific set of genes
(a Functional Gene Set, FGS) is more highly ranked on a given statistics.
In particular this functions computes a p-value for each FGS by one 
or two-sided Wilcoxon rank-sum test. Alternative user-defined functions
can also be used.

Furthermore multiple hypothesis testing correction is achieved by
applying the Benjamini and Hochberg method \cite{Benjamini1995}
as implemented in the \Rpackage{multtest} R/Bioconductor package.
Overall, this approach is conceptually analogous to Gene Set Enrichment Analysis (GSEA),
as proposed by Mootha and colleagues \cite{Mootha2003a,Subramanian2005}.

The integration can be achieved through two distinct approaches:
\begin{enumerate}
\item {\bf GSE + INTEGRATION}: Separate GSE analysisn on the 
  individual genomic platforms followed by GSE results integration;
\item {\bf INTEGRATION + GSE}: Integration of genomic data measurement 
  using a logistic model followed by GSE analysis;
\end{enumerate}

\section{RTopper data structure}
In this tutorial we demonstrate the functionality of \Rpackage{RTopper} package.
To this end we will make use of simplified data generated within 
The Cancer Genome Atlas (TCGA) project,
using Glioblastoma Multiforme (GBM) genomics data obtained from the same
patients' cohort using distinct platforms, including Differential Gene Expression (DGE),
Copy Number Variation (CNV), and Differential Methylation (DM).
This data is included with the \Rpackage{RTopper} package as the dataset \texttt{exampleData},
which consists of genomic measurements (the list \Robject{dat}) 
for 500 genes (in rows) and 95 patients (in columns) from 4 distinct platforms:
\begin{enumerate}
  \item DGE obtained using Affymetrix;
  \item DGE obtained using Agilent;
  \item CNV data generated ad Harvard;
  \item CNV data generated ad the MSKCC;
\end{enumerate}

The phenotypic class for each patient is defined in the a data.frame \Robject{pheno}
consisting of 95 rows (patients, {\it pheno\$Sample}) and 2 columns, the first being patients identifiers,
and the second variable giving the group indicator ({\it pheno\$Class}).

To load the data set type \texttt{data(exampleData)}, and to view a description of this 
data type \texttt{?exampleData}. The structure of the data is shown below:

\begin{Schunk}
\begin{Sinput}
> library(RTopper)
> data(exampleData)
> ls()
\end{Sinput}
\begin{Soutput}
[1] "dat"   "pheno"
\end{Soutput}
\begin{Sinput}
> class(dat)
\end{Sinput}
\begin{Soutput}
[1] "list"
\end{Soutput}
\begin{Sinput}
> names(dat)
\end{Sinput}
\begin{Soutput}
[1] "dat.affy"       "dat.agilent"    "dat.cnvHarvard"
[4] "dat.cnvMskcc"  
\end{Soutput}
\begin{Sinput}
> sapply(dat,class)
\end{Sinput}
\begin{Soutput}
      dat.affy    dat.agilent dat.cnvHarvard   dat.cnvMskcc 
  "data.frame"   "data.frame"   "data.frame"   "data.frame" 
\end{Soutput}
\begin{Sinput}
> sapply(dat,dim)
\end{Sinput}
\begin{Soutput}
     dat.affy dat.agilent dat.cnvHarvard dat.cnvMskcc
[1,]      500         500            500          500
[2,]       95          95             95           95
\end{Soutput}
\begin{Sinput}
> dim(pheno)
\end{Sinput}
\begin{Soutput}
[1] 95  2
\end{Soutput}
\begin{Sinput}
> str(pheno)
\end{Sinput}
\begin{Soutput}
'data.frame':	95 obs. of  2 variables:
 $ Sample: chr  "TCGA.02.0003" "TCGA.02.0007" "TCGA.02.0011" "TCGA.02.0021" ...
 $ Class : int  0 0 1 1 0 0 0 0 0 0 ...
\end{Soutput}
\end{Schunk}

In summary to perform the analysis with functions from \Rpackage{RTopper}
the genomic data used as input must be in the following format:
\begin{enumerate}
 \item {\bf Genomic measurements}: a list of data.frames, in which each list item corresponds 
  to a genomic platform, and comprises a data.frame with rows being genes and columns patients;
 \item {\bf Phenotype data}: a data.frame with 2 columns: patients and their phenotypes;
 \item The number of columns of the {\it Genomic measurements} data.frames must match the number
   of rows of the {\it Phenotype data};
 \item The same set of genes must be measured in each platform and gene labels must be stored as rownames;
\end{enumerate}

Below are shown the first 6 rows and 4  columns of each data.frame contained in \texttt{dat},
which share the same genes (shown for some of the possible combinations).
Similarly column names in the \Robject{dat} data.frames correspond to rownames of \Robject{pheno}.

\begin{Schunk}
\begin{Sinput}
> ###data structure
> lapply(dat,function(x) head(x)[,1:3])
\end{Sinput}
\begin{Soutput}
$dat.affy
      TCGA.02.0003 TCGA.02.0007 TCGA.02.0011
AACS      7.747995     7.685409     7.535661
AARS      9.381544     9.930156    10.197194
ABI1      8.173255     8.962803     9.895811
ACHE      5.127197     4.547297     5.146552
ACTC1     6.612645     5.825879     8.067945
ACTN2     6.257383     5.330557     5.842319

$dat.agilent
      TCGA.02.0003 TCGA.02.0007 TCGA.02.0011
AACS    -1.0070000   -1.1164000    -0.913000
AARS    -1.2665000   -0.8981250     0.263500
ABI1    -0.2765000    0.3356250     1.027250
ACHE     0.4403750   -0.0222500     0.115000
ACTC1    0.3641538    0.1234615     1.046692
ACTN2    4.3348000    2.2278000     3.330600

$dat.cnvHarvard
      TCGA.02.0003 TCGA.02.0007 TCGA.02.0011
AACS   -0.08273213  -0.08917331  -0.02075644
AARS   -0.10233281  -0.20620608  -0.05157664
ABI1   -0.86886659  -0.01214599   0.59307754
ACHE    0.31560002  -1.00166150  -0.14519639
ACTC1  -1.17495078  -0.26698279  -0.95662761
ACTN2  -0.11319016  -0.09657971   0.02582138

$dat.cnvMskcc
      TCGA.02.0003 TCGA.02.0007 TCGA.02.0011
AACS    -0.0383875  -0.09140000  0.008233333
AARS     0.0075600   0.02801667  0.104850000
ABI1    -0.7006900   0.21270000  0.499472727
ACHE     0.8676000  -0.23970000  0.075000000
ACTC1   -0.9779500  -0.11625000 -0.692950000
ACTN2   -0.1258571  -0.05394444  0.010200000
\end{Soutput}
\begin{Sinput}
> sum(rownames(dat[[1]])%in%rownames(dat[[2]]))
\end{Sinput}
\begin{Soutput}
[1] 500
\end{Soutput}
\begin{Sinput}
> sum(rownames(dat[[2]])%in%rownames(dat[[3]]))
\end{Sinput}
\begin{Soutput}
[1] 500
\end{Soutput}
\end{Schunk}

\subsection{Creation of Functional Gene Sets}
Functional Gene Sets (FGS) are list of genes that share a specific biological function.
Examples of FGS are genes that operate in the same signaling pathway 
({\it i.e.} Notch signaling genes), or that share the same biological function
({\it i.e.} Cell adhesion genes).
FGS can be retrieved from various database, or can be construncted {\it ad hoc}.
A convenient source of FGS are the R-Bioconductor metaData packages,
and S4 classes and methods for handling FGS are provided by the \Rpackage{GSEABase}
package. Below is shown a simple way to extract FGS from the human genome
metaData package \Rpackage{org.Hs.eg.db}.
As a general rule the name of the metaData package, without the \texttt{.db} extension,
can be used a function to see the content of the package, as shown below:

\begin{Schunk}
\begin{Sinput}
> library(org.Hs.eg.db)
> org.Hs.eg()
\end{Sinput}
\begin{Soutput}
Quality control information for org.Hs.eg:


This package has the following mappings:

org.Hs.egACCNUM has 29598 mapped keys (of 45341 keys)
org.Hs.egACCNUM2EG has 665642 mapped keys (of 665642 keys)
org.Hs.egALIAS2EG has 112379 mapped keys (of 112379 keys)
org.Hs.egCHR has 44908 mapped keys (of 45341 keys)
org.Hs.egCHRLENGTHS has 93 mapped keys (of 93 keys)
org.Hs.egCHRLOC has 22085 mapped keys (of 45341 keys)
org.Hs.egCHRLOCEND has 22085 mapped keys (of 45341 keys)
org.Hs.egENSEMBL has 19542 mapped keys (of 45341 keys)
org.Hs.egENSEMBL2EG has 19948 mapped keys (of 19948 keys)
org.Hs.egENSEMBLPROT has 19476 mapped keys (of 45341 keys)
org.Hs.egENSEMBLPROT2EG has 83286 mapped keys (of 83286 keys)
org.Hs.egENSEMBLTRANS has 19529 mapped keys (of 45341 keys)
org.Hs.egENSEMBLTRANS2EG has 121860 mapped keys (of 121860 keys)
org.Hs.egENZYME has 2232 mapped keys (of 45341 keys)
org.Hs.egENZYME2EG has 975 mapped keys (of 975 keys)
org.Hs.egGENENAME has 45341 mapped keys (of 45341 keys)
org.Hs.egGO has 17814 mapped keys (of 45341 keys)
org.Hs.egGO2ALLEGS has 14407 mapped keys (of 14407 keys)
org.Hs.egGO2EG has 10981 mapped keys (of 10981 keys)
org.Hs.egMAP has 35270 mapped keys (of 45341 keys)
org.Hs.egMAP2EG has 2589 mapped keys (of 2589 keys)
org.Hs.egOMIM has 14856 mapped keys (of 45341 keys)
org.Hs.egOMIM2EG has 17661 mapped keys (of 17661 keys)
org.Hs.egPATH has 5891 mapped keys (of 45341 keys)
org.Hs.egPATH2EG has 229 mapped keys (of 229 keys)
org.Hs.egPFAM has 22269 mapped keys (of 45341 keys)
org.Hs.egPMID has 30319 mapped keys (of 45341 keys)
org.Hs.egPMID2EG has 302146 mapped keys (of 302146 keys)
org.Hs.egPROSITE has 22269 mapped keys (of 45341 keys)
org.Hs.egREFSEQ has 28137 mapped keys (of 45341 keys)
org.Hs.egREFSEQ2EG has 86715 mapped keys (of 86715 keys)
org.Hs.egSYMBOL has 45341 mapped keys (of 45341 keys)
org.Hs.egSYMBOL2EG has 45329 mapped keys (of 45329 keys)
org.Hs.egUCSCKG has 20522 mapped keys (of 45341 keys)
org.Hs.egUNIGENE has 23787 mapped keys (of 45341 keys)
org.Hs.egUNIGENE2EG has 24453 mapped keys (of 24453 keys)
org.Hs.egUNIPROT has 19163 mapped keys (of 45341 keys)


Additional Information about this package:

DB schema: HUMAN_DB
DB schema version: 2.1
Organism: Homo sapiens
Date for NCBI data: 2011-Mar16
Date for GO data: 20110312
Date for KEGG data: 2011-Mar15
Date for Golden Path data: 2010-Mar22
Date for IPI data: 2011-Feb18
Date for Ensembl data: 2011-Feb2
\end{Soutput}
\end{Schunk}

For instance the \texttt{org.Hs.egGO2ALLEGS} environment contains the mapping
of all ENTREZ Gene identifiers to the {\bf Gene Ontology Terms} \cite{Ashburner2000},
while \texttt{org.Hs.egPATH2EG} maps the identifiers to {\bf KEGG} 
pathways \cite{Kanehisa2004}.
The corresponding lists of FGS can be retrieve from the corresponding environments
using the the R command \texttt{as.list()}, as shown below for KEGG and GO:

\begin{Schunk}
\begin{Sinput}
> kegg <- as.list(org.Hs.egPATH2EG)
> go <- as.list(org.Hs.egGO2ALLEGS)
> length(kegg)
\end{Sinput}
\begin{Soutput}
[1] 229
\end{Soutput}
\begin{Sinput}
> length(go)
\end{Sinput}
\begin{Soutput}
[1] 14407
\end{Soutput}
\begin{Sinput}
> str(kegg[1:5])
\end{Sinput}
\begin{Soutput}
List of 5
 $ 00010: chr [1:65] "124" "125" "126" "127" ...
 $ 00020: chr [1:30] "47" "48" "50" "1431" ...
 $ 00030: chr [1:27] "226" "229" "230" "2203" ...
 $ 00040: chr [1:32] "217" "219" "224" "231" ...
 $ 00051: chr [1:36] "226" "229" "230" "231" ...
\end{Soutput}
\begin{Sinput}
> names(kegg)[1:5]
\end{Sinput}
\begin{Soutput}
[1] "00010" "00020" "00030" "00040" "00051"
\end{Soutput}
\begin{Sinput}
> str(go[1:5])
\end{Sinput}
\begin{Soutput}
List of 5
 $ GO:0000002: Named chr [1:11] "291" "1763" "1890" "4358" ...
  ..- attr(*, "names")= chr [1:11] "TAS" "IDA" "TAS" "IMP" ...
 $ GO:0000003: Named chr [1:1138] "18" "49" "49" "49" ...
  ..- attr(*, "names")= chr [1:1138] "IEA" "IEA" "IMP" "ISS" ...
 $ GO:0000012: Named chr [1:8] "3981" "7141" "7515" "54840" ...
  ..- attr(*, "names")= chr [1:8] "IDA" "IDA" "IEA" "IDA" ...
 $ GO:0000018: Named chr [1:29] "641" "641" "958" "2956" ...
  ..- attr(*, "names")= chr [1:29] "IEA" "IMP" "IEA" "IDA" ...
 $ GO:0000019: Named chr [1:4] "641" "4292" "4361" "10111"
  ..- attr(*, "names")= chr [1:4] "IEA" "IEA" "TAS" "IDA"
\end{Soutput}
\begin{Sinput}
> names(go)[1:5]
\end{Sinput}
\begin{Soutput}
[1] "GO:0000002" "GO:0000003" "GO:0000012" "GO:0000018" "GO:0000019"
\end{Soutput}
\end{Schunk}

In the \Robject{kegg} list genes are identified by their ENTREZ Gene identifiers,
while in the \Robject{dat} genes are identified by their Gene Symbol.
Below is an example of the code that can be used to perform the identifiers conversion,
using only a subset of KEGG and GO FGS:

\begin{Schunk}
\begin{Sinput}
> kegg <- lapply(kegg[sample(1:length(kegg),5)],function(x) unique(unlist(mget(x,org.Hs.egSYMBOL))))
> go <- lapply(go[sample(1:length(go),5)],function(x) unique(unlist(mget(x,org.Hs.egSYMBOL))))
> str(kegg)
\end{Sinput}
\begin{Soutput}
List of 5
 $ 00250: chr [1:32] "ABAT" "ADSL" "ADSS" "AGXT" ...
 $ 00061: chr [1:6] "ACACA" "ACACB" "FASN" "MCAT" ...
 $ 04620: chr [1:102] "AKT1" "AKT2" "CASP8" "CD14" ...
 $ 04630: chr [1:155] "AKT1" "AKT2" "CCND1" "BCL2L1" ...
 $ 00980: chr [1:71] "ADH1A" "ADH1B" "ADH1C" "ADH4" ...
\end{Soutput}
\begin{Sinput}
> str(go)
\end{Sinput}
\begin{Soutput}
List of 5
 $ GO:0004915: chr [1:2] "IL6R" "IL6ST"
 $ GO:0051902: chr [1:2] "BCL2" "IFI6"
 $ GO:0005732: chr [1:13] "FBL" "NHP2L1" "RRP9" "SNRNP40" ...
 $ GO:2000191: chr [1:14] "AKT1" "AKT2" "IL1B" "OXT" ...
 $ GO:0004382: chr "ENTPD5"
\end{Soutput}
\end{Schunk}

Finally, it is also possible to annotate FGS, mapping pathways identifiers to pathway names,
as shown below for KEGG, using the \Rpackage{KEGG.db}.

\begin{Schunk}
\begin{Sinput}
> library(KEGG.db)
> KEGG()
\end{Sinput}
\begin{Soutput}
Quality control information for KEGG:


This package has the following mappings:

KEGGENZYMEID2GO has 3922 mapped keys (of 3922 keys)
KEGGEXTID2PATHID has 75100 mapped keys (of 75100 keys)
KEGGGO2ENZYMEID has 3976 mapped keys (of 3976 keys)
KEGGPATHID2EXTID has 3152 mapped keys (of 3152 keys)
KEGGPATHID2NAME has 390 mapped keys (of 390 keys)
KEGGPATHNAME2ID has 390 mapped keys (of 390 keys)


Additional Information about this package:

DB schema: KEGG_DB
DB schema version: 2.1
Date for KEGG data: 2011-Mar15
\end{Soutput}
\begin{Sinput}
> names(kegg) <- paste(names(kegg),unlist(mget(names(kegg),KEGGPATHID2NAME)),sep=".")
> names(kegg)
\end{Sinput}
\begin{Soutput}
[1] "00250.Alanine, aspartate and glutamate metabolism" 
[2] "00061.Fatty acid biosynthesis"                     
[3] "04620.Toll-like receptor signaling pathway"        
[4] "04630.Jak-STAT signaling pathway"                  
[5] "00980.Metabolism of xenobiotics by cytochrome P450"
\end{Soutput}
\end{Schunk}

Similarly GO Terms can be retrieved from the \Rpackage{GO.db}
(please refer to the vignettes of the corresponding packages for details).

\begin{Schunk}
\begin{Sinput}
> library(GO.db)
> GO()
\end{Sinput}
\begin{Soutput}
Quality control information for GO:


This package has the following mappings:

GOBPANCESTOR has 20443 mapped keys (of 20443 keys)
GOBPCHILDREN has 11741 mapped keys (of 20443 keys)
GOBPOFFSPRING has 11741 mapped keys (of 20443 keys)
GOBPPARENTS has 20443 mapped keys (of 20443 keys)
GOCCANCESTOR has 2811 mapped keys (of 2811 keys)
GOCCCHILDREN has 963 mapped keys (of 2811 keys)
GOCCOFFSPRING has 963 mapped keys (of 2811 keys)
GOCCPARENTS has 2811 mapped keys (of 2811 keys)
GOMFANCESTOR has 9000 mapped keys (of 9000 keys)
GOMFCHILDREN has 1834 mapped keys (of 9000 keys)
GOMFOFFSPRING has 1834 mapped keys (of 9000 keys)
GOMFPARENTS has 9000 mapped keys (of 9000 keys)
GOOBSOLETE has 1545 mapped keys (of 1545 keys)
GOTERM has 32255 mapped keys (of 32255 keys)


Additional Information about this package:

DB schema: GO_DB
DB schema version: 2.1
Date for GO data: 20110312
\end{Soutput}
\begin{Sinput}
> names(go) <- paste(names(go),Term(names(go)),sep=".")
> names(go)
\end{Sinput}
\begin{Soutput}
[1] "GO:0004915.interleukin-6 receptor activity"                    
[2] "GO:0051902.negative regulation of mitochondrial depolarization"
[3] "GO:0005732.small nucleolar ribonucleoprotein complex"          
[4] "GO:2000191.regulation of fatty acid transport"                 
[5] "GO:0004382.guanosine-diphosphatase activity"                   
\end{Soutput}
\end{Schunk}

Finally we can be combine the two FGS collections into a named list for further used
in GSE analysis (see below).

\begin{Schunk}
\begin{Sinput}
> fgsList <- list(go=go,kegg=kegg)
\end{Sinput}
\end{Schunk}

\section{Data analysis with RTopper}
To compute gene-to-phenotype association scores the first step required 
is the convertion of the data into a list, where each list item corresponds to a gene, 
and comprises a data.frame with the rows being patients, and columns being measurements 
for each data type, along with the class phenotype ({\it the response}).
Importantly each element of the list with the data should have the same genes and patients.

The \Rfunction{convertToDr} function is used to make such conversion.
Below is a short description of the arguments to this function:
\begin{itemize}
  \item \Rfunarg{dataIntersection}: a list of data.frames containing the same set of patients(columns) and genes (rows)
  \item \Rfunarg{response}: a data.frame indicating patients' phenotypic class;
  \item \Rfunarg{nPlatforms}: the number of platforms;
\end{itemize}

This can be achieved as follows using our examples data:

\begin{Schunk}
\begin{Sinput}
> dataDr <- convertToDr(dat, pheno, 4)
> class(dataDr)
\end{Sinput}
\begin{Soutput}
[1] "list"
\end{Soutput}
\begin{Sinput}
> length(dataDr)
\end{Sinput}
\begin{Soutput}
[1] 500
\end{Soutput}
\begin{Sinput}
> names(dataDr)[1:5]
\end{Sinput}
\begin{Soutput}
[1] "AACS"  "AARS"  "ABI1"  "ACHE"  "ACTC1"
\end{Soutput}
\begin{Sinput}
> str(dataDr[1:2])
\end{Sinput}
\begin{Soutput}
List of 2
 $ AACS:'data.frame':	95 obs. of  5 variables:
  ..$ dat.affy      : num [1:95] 7.75 7.69 7.54 7.3 7.01 ...
  ..$ dat.agilent   : num [1:95] -1.007 -1.116 -0.913 -1.061 -1.775 ...
  ..$ dat.cnvHarvard: num [1:95] -0.0827 -0.0892 -0.0208 -0.1811 -0.0625 ...
  ..$ dat.cnvMskcc  : num [1:95] -0.03839 -0.0914 0.00823 0.03456 0.0573 ...
  ..$ response      : int [1:95] 0 0 1 1 0 0 0 0 0 0 ...
 $ AARS:'data.frame':	95 obs. of  5 variables:
  ..$ dat.affy      : num [1:95] 9.38 9.93 10.2 9.54 9.37 ...
  ..$ dat.agilent   : num [1:95] -1.266 -0.898 0.264 -0.599 -1.437 ...
  ..$ dat.cnvHarvard: num [1:95] -0.1023 -0.2062 -0.0516 -0.0923 -0.1199 ...
  ..$ dat.cnvMskcc  : num [1:95] 0.00756 0.02802 0.10485 0.0841 0.12262 ...
  ..$ response      : int [1:95] 0 0 1 1 0 0 0 0 0 0 ...
\end{Soutput}
\end{Schunk}

It is now possible to compute gene-to-phenotype association scores, using as input the
gene-centered list produced by \Rfunction{convertToDr}.
Therefore the \Rfunction{computeDrStat} function assumes that each gene-centered 
data.frame contains a column (the last one) called \texttt{'response'}, 
as created by the \Rfunction{convertToDr}.
Below is a short description of the arguments to this function:

\begin{itemize}
  \item \Rfunarg{data}: a list of data.frames, one for each gene analyzed, contining the the genomic
    measurements from all platforms (by column) for all the patients (by row), along with the phenotypic
    response;
  \item \Rfunarg{columns}: a numeric vector indicating column indexes corresponding the genomic measurements
    to be used for computing the gene-to-phenotype association scores;
    the default is \texttt{columns = c(1:(ncol(data) - 1))}, assuming the phenotypic response to be the last
    column;
  \item \Rfunarg{method}: the method used to compute the association score;
  \item \Rfunarg{integrate}: logical, whether an integrated gene-to-phenotype score should be computed,
    or separate scores for each platform/data sets specified by \Rfunarg{columns};
  \end{itemize}

In the current implementation of the \Rpackage{RTopper} there are three methods for computing 
gene-to-phenotype association scores:
\begin{enumerate}
  \item \Rmethod{dev}: this approach computes the score as the difference of deviances
    (as described in Tyekucheva et al, manuscript under review \cite{Tyekucheva2011});
  \item \Rmethod{aic}: this approach computes the score as the Akaike information criterion
    for model selection;
  \item \Rmethod{bic}: this approach computes the score as the penalized likelihood ratio;
\end{enumerate}

\subsection{Integrated Gene-to-Phenotype score computation}
This approach first integrates genomic data across platform, and subsequently
perform GSE to identify the FGS most strongly associated with the integrated score.
Below is an example of application to compute the gene-to-phenotype association
scores for 4 data type simultaneously:

\begin{Schunk}
\begin{Sinput}
> bicStatInt <- computeDrStat(dataDr, columns = c(1:4), method="bic", integrate = TRUE)
> names(bicStatInt)
\end{Sinput}
\begin{Soutput}
[1] "integrated"
\end{Soutput}
\begin{Sinput}
> str(bicStatInt)
\end{Sinput}
\begin{Soutput}
List of 1
 $ integrated: Named num [1:500] -11.43 -15.93 -8.85 -13.52 -7.26 ...
  ..- attr(*, "names")= chr [1:500] "AACS" "AARS" "ABI1" "ACHE" ...
\end{Soutput}
\end{Schunk}

%%%####HERE

\subsection{Separate Gene-to-Phenotype score computation}
This approach first computes computes gene-to-phenotype score separately for each platform,
uses the scores to perform separate GSE analysis in each platform for identifying the FGS most 
strongly associated with the score, and finally integrates the results from GSE analysis,
Below is an example of this approach:

\begin{Schunk}
\begin{Sinput}
> bicStatSep <- computeDrStat(dataDr, columns = c(1:4), method="bic", integrate = FALSE)
> names(bicStatSep)
\end{Sinput}
\begin{Soutput}
[1] "dat.affy"       "dat.agilent"    "dat.cnvHarvard"
[4] "dat.cnvMskcc"  
\end{Soutput}
\begin{Sinput}
> str(bicStatSep)
\end{Sinput}
\begin{Soutput}
List of 4
 $ dat.affy      : Named num [1:500] 0.545 -4.269 -2.334 -4.471 -3.625 ...
  ..- attr(*, "names")= chr [1:500] "AACS" "AARS" "ABI1" "ACHE" ...
 $ dat.agilent   : Named num [1:500] -3.57 -4.5 -3.66 -4.52 -1.05 ...
  ..- attr(*, "names")= chr [1:500] "AACS" "AARS" "ABI1" "ACHE" ...
 $ dat.cnvHarvard: Named num [1:500] -4.49 -3.64 3.13 -3.26 -2.57 ...
  ..- attr(*, "names")= chr [1:500] "AACS" "AARS" "ABI1" "ACHE" ...
 $ dat.cnvMskcc  : Named num [1:500] -4.53 -4.48 2.1 -2.55 -4.25 ...
  ..- attr(*, "names")= chr [1:500] "AACS" "AARS" "ABI1" "ACHE" ...
\end{Soutput}
\end{Schunk}

\subsection{Gene Set Enrichment using integrated and separate score}
After the gene-to-phenotype scores have been obtained it is possible to perform
a GSE analysis. To this end we will use the \Rfunction{runBatchGSE} function,
as shown below. This function enables to perform GSE analysis over multiple collections
of FGS, and over multiple ranking statistics.
In the current implementation of the \Rfunction{runBatchGSE} the default is
performing the enrichment analysis using the \Rfunction{geneSetTest} function 
from the \Rpackage{limma} package, and most of the arguments passed to 
\Rpackage{runBatchGSE} are indeed passed to \Rfunction{geneSetTest}
(see the relative help for the details).

As an alternative the user can also define his own function to test for FGS enrichment,
passing the selection of genes within the FGS and the ranking ranking statistics
in the same way as done for \Rfunction{geneSetTest}.
In this tutorial we apply \Rfunction{geneSetTest} in order to perform a
Wilcoxon rank-sum test, using the absolute value of the gene-to-phenotype scores 
as the ranking statistics.

\begin{Schunk}
\begin{Sinput}
> args(runBatchGSE)
\end{Sinput}
\begin{Soutput}
function (dataList, fgsList, ...) 
NULL
\end{Soutput}
\end{Schunk}

Below a short description of the arguments that can be passed to this function:
\begin{itemize}
 \item \Rfunarg{dataList}: a list containing gene-to-phenotype scores to be used
   as ranking statistics in the GSE analysis;
 \item \Rfunarg{fgsList}: a list of FGS collection, in which each element is a list of character vectors,
   one for each gene set;
 \item \Rfunarg{...}: any other argument to be passed to lower level functions, including 
   the lower level  enrichment function to be used (like the \Rfunction{geneSetTest} function
   from the \Rpackage{limma} package, which is used as the default);
 \item \Rfunarg{absolute}: logical specifying whether the absolute values of the ranking statistics 
   should be used in the test (the default being TRUE);
 \item \Rfunarg{gseFunc}: a function to perform GSE analysis, when not specified (the default) the
   \Rfunction{geneSetTest} from the \Rpackage{limma} package is used. When a function is specified,
   the membership of the analyzed genes to a FGS, and the ranking statistics must be defined in the
   same way this is done for \Rfunction{geneSetTest}, and the new function must
   return an integer (usually a p-value) (see the help for \Rfunction{geneSetTest}
   in the \Rpackage{limma} package)
 \end{itemize}

Below are few examples to perform Wilcoxon rank-sum test over multiple FGS collections,
and over multiple ranking statistics, usin the \Rfunction{runBatchGSE}.
To this end we will use the {\bf KEGG} and {\bf GO} collections created above,
and the separate and integrated gene-to-phenotype scores computed using the
\Rfunction{computeDrStat}.
The output of this function is a named list of lists, containing an element for each
ranking statistics considered in the input. Each one of these elements, in turn,
is another list, containing the GSE results for each collection sets.
In the examples below we will therefore obtain a list of length one in the case
ot the integrated gene-to-phenotype score, and a list of length four
(on element for each genomic platform) in the case of the separate scores.
For all the rankings we will obtain GSE result for both the collections of FGS.

\subsection{INTEGRATION + GSE}
The integrated gene-to-phenotype scores we have computed can be used to
perform a GSE analysis. Below are reported few examples, using the default
options, as well as passing several specific arguments to \Rfunction{geneSetTest}
(see the relative help for details).

\subsubsection{One-sided Wilcoxon rank-sum test using absolute ranking statistics}
This can be accomplished by calling the \Rfunction{runBatchGSE} with default
values, or by specifying each argument, as shown below:

\begin{Schunk}
\begin{Sinput}
> gseABS.int <- runBatchGSE(dataList=bicStatInt, fgsList=fgsList)
> gseABS.int <- runBatchGSE(dataList=bicStatInt, fgsList=fgsList,
+ 				 absolute=TRUE, type="f", alternative="mixed")
\end{Sinput}
\end{Schunk}

\subsubsection{One-sided Wilcoxon rank-sum test using signed ranking statistics}
When the signed ranking statistics has a sign, it is possible to perform a one-sided
test assensing both tails separately, as well as a two-sided test.
This can be accomplished by passing the corresponding arguments 
to \Rfunction{runBatchGSE}, as shown below:

\begin{Schunk}
\begin{Sinput}
> gseUP.int <- runBatchGSE(dataList=bicStatInt, fgsList=fgsList,
+ 				 absolute=FALSE, type="t", alternative="up")
> gseDW.int <- runBatchGSE(dataList=bicStatInt, fgsList=fgsList,
+ 				 absolute=FALSE, type="t", alternative="down")
> gseBOTH.int <- runBatchGSE(dataList=bicStatInt, fgsList=fgsList,
+ 				 absolute=FALSE, type="t", alternative="either")
\end{Sinput}
\end{Schunk}

\subsubsection{Performing a simulation-based GSE test}
It is also possible to perform an enrichment analysis comparing each FGS
to randomly selected gene lists of the same size of the FGS.
In this case  the p-value is computed by simulation as the proportion 
of times the mean of the statistics in the FGS is smaller (or larger) than in the 
\Rfunarg{nsim} random simulated sets of genes.

\begin{Schunk}
\begin{Sinput}
> gseABSsim.int <- runBatchGSE(dataList=bicStatInt, fgsList=fgsList,
+ 				    absolute=TRUE, type="f", alternative="mixed",
+ 				    ranks.only=FALSE, nsim=1000)
> gseUPsim.int <- runBatchGSE(dataList=bicStatInt, fgsList=fgsList,
+ 				    absolute=FALSE, type="t", alternative="up",
+ 				    ranks.only=FALSE, nsim=1000)
\end{Sinput}
\end{Schunk}

Results from this analysis are named lists of lists, as shown below:
\begin{Schunk}
\begin{Sinput}
> str(gseUP.int)
\end{Sinput}
\begin{Soutput}
List of 1
 $ integrated:List of 2
  ..$ go  : Named num [1:5] NA NA NA 0.407 NA
  .. ..- attr(*, "names")= chr [1:5] "GO:0004915.interleukin-6 receptor activity" "GO:0051902.negative regulation of mitochondrial depolarization" "GO:0005732.small nucleolar ribonucleoprotein complex" "GO:2000191.regulation of fatty acid transport" ...
  ..$ kegg: Named num [1:5] NA NA 0.179 0.263 0.957
  .. ..- attr(*, "names")= chr [1:5] "00250.Alanine, aspartate and glutamate metabolism" "00061.Fatty acid biosynthesis" "04620.Toll-like receptor signaling pathway" "04630.Jak-STAT signaling pathway" ...
\end{Soutput}
\begin{Sinput}
> gseABSsim.int
\end{Sinput}
\begin{Soutput}
$integrated
$integrated$go
                    GO:0004915.interleukin-6 receptor activity 
                                                            NA 
GO:0051902.negative regulation of mitochondrial depolarization 
                                                            NA 
          GO:0005732.small nucleolar ribonucleoprotein complex 
                                                            NA 
                 GO:2000191.regulation of fatty acid transport 
                                                     0.5544456 
                   GO:0004382.guanosine-diphosphatase activity 
                                                            NA 

$integrated$kegg
 00250.Alanine, aspartate and glutamate metabolism 
                                                NA 
                     00061.Fatty acid biosynthesis 
                                                NA 
        04620.Toll-like receptor signaling pathway 
                                        0.64635365 
                  04630.Jak-STAT signaling pathway 
                                        0.65234765 
00980.Metabolism of xenobiotics by cytochrome P450 
                                        0.04695305 
\end{Soutput}
\end{Schunk}

\subsubsection{Passsing alternative enrichment functions to  \Rfunction{runBatchGSE} }
Below is show how to define and pass alternative enrichment functions 
to \Rfunction{runBatchGSE}.
We will first show how to use the \Rpackage{limma} \Rfunction{wilcoxGST} function,
which is a synonym for \Rfunction{geneSetTest} using \Rfunarg{ranks.only=TRUE}
and \Rfunarg{type="t"}.

\begin{Schunk}
\begin{Sinput}
> library(limma)
> gseUP.int.2 <- runBatchGSE(dataList=bicStatInt, fgsList=fgsList,
+ 				 absolute=FALSE, gseFunc=wilcoxGST, alternative="up")
\end{Sinput}
\end{Schunk}

As shown below this approach will return the same results
obtained with \Rfunction{geneSetTest} passing appropriate arguments.

\begin{Schunk}
\begin{Sinput}
> str(gseUP.int.2)
\end{Sinput}
\begin{Soutput}
List of 1
 $ integrated:List of 2
  ..$ go  : Named num [1:5] NA NA NA 0.407 NA
  .. ..- attr(*, "names")= chr [1:5] "GO:0004915.interleukin-6 receptor activity" "GO:0051902.negative regulation of mitochondrial depolarization" "GO:0005732.small nucleolar ribonucleoprotein complex" "GO:2000191.regulation of fatty acid transport" ...
  ..$ kegg: Named num [1:5] NA NA 0.179 0.263 0.957
  .. ..- attr(*, "names")= chr [1:5] "00250.Alanine, aspartate and glutamate metabolism" "00061.Fatty acid biosynthesis" "04620.Toll-like receptor signaling pathway" "04630.Jak-STAT signaling pathway" ...
\end{Soutput}
\begin{Sinput}
> all(gseUP.int.2$go==gseUP.int$go)
\end{Sinput}
\begin{Soutput}
[1] TRUE
\end{Soutput}
\end{Schunk}

We can finally also pass any new user-defined enrichment function,
provided that the arguments are passed in the same way as with
\Rfunction{geneSetTest}, as shown below using the Fisher's exact test,
and a  threshold for defining the list of differentially expressed genes.

\begin{Schunk}
\begin{Sinput}
> gseFunc <- function (selected, statistics, threshold) {
+ 	diffExpGenes <- statistics > threshold
+ 	tab <- table(diffExpGenes, selected)
+ 	pVal <- fisher.test(tab)[["p.value"]]
+ 	}
> gseUP.int.3 <- runBatchGSE(dataList=bicStatInt, fgsList=fgsList,
+ 				 absolute=FALSE, gseFunc=gseFunc, threshold=7.5)
\end{Sinput}
\end{Schunk}

As shown below this approach will test for over-represtation of the
a specific gene set within the genes defined as differentially expressed
(in our example the genes showing an integrated association score
larger than 7.5). Results are somewhat comparable to what obtained
using the Wilcoxon rank-sum test.

\begin{Schunk}
\begin{Sinput}
> str(gseUP.int.3)
\end{Sinput}
\begin{Soutput}
List of 1
 $ integrated:List of 2
  ..$ go  : Named num [1:5] NA NA NA 1 NA
  .. ..- attr(*, "names")= chr [1:5] "GO:0004915.interleukin-6 receptor activity" "GO:0051902.negative regulation of mitochondrial depolarization" "GO:0005732.small nucleolar ribonucleoprotein complex" "GO:2000191.regulation of fatty acid transport" ...
  ..$ kegg: Named num [1:5] NA NA 1 1 1
  .. ..- attr(*, "names")= chr [1:5] "00250.Alanine, aspartate and glutamate metabolism" "00061.Fatty acid biosynthesis" "04620.Toll-like receptor signaling pathway" "04630.Jak-STAT signaling pathway" ...
\end{Soutput}
\begin{Sinput}
> data.frame(fisher=gseUP.int.3$integrated$kegg,wilcoxon=gseUP.int$integrated$kegg)
\end{Sinput}
\begin{Soutput}
                                                   fisher  wilcoxon
00250.Alanine, aspartate and glutamate metabolism      NA        NA
00061.Fatty acid biosynthesis                          NA        NA
04620.Toll-like receptor signaling pathway              1 0.1790416
04630.Jak-STAT signaling pathway                        1 0.2626014
00980.Metabolism of xenobiotics by cytochrome P450      1 0.9572340
\end{Soutput}
\end{Schunk}

\subsection{GSE + INTEGRATION}
The individual gene-to-phenotype scores computed for each platform
can be similarly used to perform separate GSE analyses for each considered
genomic platform, applying the same code and functions used to perform
GSE analysis in the {\bf INTEGRATION + GSE} approach above.

\begin{Schunk}
\begin{Sinput}
> gseABS.sep <- runBatchGSE(dataList=bicStatSep, fgsList=fgsList)
\end{Sinput}
\end{Schunk}

This step of GSE analysis on separate platform is then followed 
by GSE results integration, which is achieved using the 
\Rfunction{combineGSE} function, which summarizes the
individual p-values from the  tests.
To this end different methods are available, including the computation
of the geometric or arithmetic  means, the use of the median, 
the selection of the minimun or the maximum p-value, and
the random selection (respectively \Rfunarg{geometricMean},
\Rfunarg{mean}, \Rfunarg{median}, \Rfunarg{min}, \Rfunarg{max},
and \Rfunarg{random}). Few examples are shown below:

\begin{Schunk}
\begin{Sinput}
> gseABS.geoMean.sep <- combineGSE(gseABS.sep, method="geometricMean")
> gseABS.max.sep <- combineGSE(gseABS.sep, method="max")
\end{Sinput}
\end{Schunk}

Also in this case the results from the combination are named lists of lists,
as shown below:

\begin{Schunk}
\begin{Sinput}
> names(gseABS.sep)
\end{Sinput}
\begin{Soutput}
[1] "dat.affy"       "dat.agilent"    "dat.cnvHarvard"
[4] "dat.cnvMskcc"  
\end{Soutput}
\begin{Sinput}
> str(gseABS.sep)
\end{Sinput}
\begin{Soutput}
List of 4
 $ dat.affy      :List of 2
  ..$ go  : Named num [1:5] NA NA NA 0.55 NA
  .. ..- attr(*, "names")= chr [1:5] "GO:0004915.interleukin-6 receptor activity" "GO:0051902.negative regulation of mitochondrial depolarization" "GO:0005732.small nucleolar ribonucleoprotein complex" "GO:2000191.regulation of fatty acid transport" ...
  ..$ kegg: Named num [1:5] NA NA 0.387 0.942 0.138
  .. ..- attr(*, "names")= chr [1:5] "00250.Alanine, aspartate and glutamate metabolism" "00061.Fatty acid biosynthesis" "04620.Toll-like receptor signaling pathway" "04630.Jak-STAT signaling pathway" ...
 $ dat.agilent   :List of 2
  ..$ go  : Named num [1:5] NA NA NA 0.341 NA
  .. ..- attr(*, "names")= chr [1:5] "GO:0004915.interleukin-6 receptor activity" "GO:0051902.negative regulation of mitochondrial depolarization" "GO:0005732.small nucleolar ribonucleoprotein complex" "GO:2000191.regulation of fatty acid transport" ...
  ..$ kegg: Named num [1:5] NA NA 0.87954 0.53002 0.00854
  .. ..- attr(*, "names")= chr [1:5] "00250.Alanine, aspartate and glutamate metabolism" "00061.Fatty acid biosynthesis" "04620.Toll-like receptor signaling pathway" "04630.Jak-STAT signaling pathway" ...
 $ dat.cnvHarvard:List of 2
  ..$ go  : Named num [1:5] NA NA NA 0.118 NA
  .. ..- attr(*, "names")= chr [1:5] "GO:0004915.interleukin-6 receptor activity" "GO:0051902.negative regulation of mitochondrial depolarization" "GO:0005732.small nucleolar ribonucleoprotein complex" "GO:2000191.regulation of fatty acid transport" ...
  ..$ kegg: Named num [1:5] NA NA 0.977 0.16 0.899
  .. ..- attr(*, "names")= chr [1:5] "00250.Alanine, aspartate and glutamate metabolism" "00061.Fatty acid biosynthesis" "04620.Toll-like receptor signaling pathway" "04630.Jak-STAT signaling pathway" ...
 $ dat.cnvMskcc  :List of 2
  ..$ go  : Named num [1:5] NA NA NA 0.566 NA
  .. ..- attr(*, "names")= chr [1:5] "GO:0004915.interleukin-6 receptor activity" "GO:0051902.negative regulation of mitochondrial depolarization" "GO:0005732.small nucleolar ribonucleoprotein complex" "GO:2000191.regulation of fatty acid transport" ...
  ..$ kegg: Named num [1:5] NA NA 0.934 0.721 0.527
  .. ..- attr(*, "names")= chr [1:5] "00250.Alanine, aspartate and glutamate metabolism" "00061.Fatty acid biosynthesis" "04620.Toll-like receptor signaling pathway" "04630.Jak-STAT signaling pathway" ...
\end{Soutput}
\begin{Sinput}
> str(gseABS.geoMean.sep)
\end{Sinput}
\begin{Soutput}
List of 1
 $ combinedScore:List of 2
  ..$ go  : Named num [1:5] NA NA NA 0.335 NA
  .. ..- attr(*, "names")= chr [1:5] "GO:0004915.interleukin-6 receptor activity" "GO:0051902.negative regulation of mitochondrial depolarization" "GO:0005732.small nucleolar ribonucleoprotein complex" "GO:2000191.regulation of fatty acid transport" ...
  ..$ kegg: Named num [1:5] NA NA 0.747 0.49 0.154
  .. ..- attr(*, "names")= chr [1:5] "00250.Alanine, aspartate and glutamate metabolism" "00061.Fatty acid biosynthesis" "04620.Toll-like receptor signaling pathway" "04630.Jak-STAT signaling pathway" ...
\end{Soutput}
\begin{Sinput}
> gseABS.geoMean.sep
\end{Sinput}
\begin{Soutput}
$combinedScore
$combinedScore$go
                    GO:0004915.interleukin-6 receptor activity 
                                                            NA 
GO:0051902.negative regulation of mitochondrial depolarization 
                                                            NA 
          GO:0005732.small nucleolar ribonucleoprotein complex 
                                                            NA 
                 GO:2000191.regulation of fatty acid transport 
                                                     0.3346236 
                   GO:0004382.guanosine-diphosphatase activity 
                                                            NA 

$combinedScore$kegg
 00250.Alanine, aspartate and glutamate metabolism 
                                                NA 
                     00061.Fatty acid biosynthesis 
                                                NA 
        04620.Toll-like receptor signaling pathway 
                                         0.7466514 
                  04630.Jak-STAT signaling pathway 
                                         0.4901592 
00980.Metabolism of xenobiotics by cytochrome P450 
                                         0.1537194 
\end{Soutput}
\end{Schunk}

\subsection{Multiple testing correction}
Finally the \Rfunction{adjustPvalGSE} enables to adjust the
p-values computed by the \Rfunction{runBatchGSE}.
This functions is an interface to the \Rfunction{mt.rawp2adjp}
function from the \Rpackage{multtest} package.

\begin{Schunk}
\begin{Sinput}
> gseABS.int.BH <- adjustPvalGSE(gseABS.int)
> gseABS.int.holm <- adjustPvalGSE(gseABS.int, proc = "Holm")
\end{Sinput}
\end{Schunk}

Also in this case the results after the adjustment are named lists of lists,
as shown below:

\begin{Schunk}
\begin{Sinput}
> names(gseABS.int.BH)
\end{Sinput}
\begin{Soutput}
[1] "integrated"
\end{Soutput}
\begin{Sinput}
> names(gseABS.int.holm)
\end{Sinput}
\begin{Soutput}
[1] "integrated"
\end{Soutput}
\begin{Sinput}
> str(gseABS.int.BH)
\end{Sinput}
\begin{Soutput}
List of 1
 $ integrated:List of 2
  ..$ go  : num [1:5, 1:2] NA NA NA 0.598 NA ...
  .. ..- attr(*, "dimnames")=List of 2
  .. .. ..$ : chr [1:5] "GO:0004915.interleukin-6 receptor activity" "GO:0051902.negative regulation of mitochondrial depolarization" "GO:0005732.small nucleolar ribonucleoprotein complex" "GO:2000191.regulation of fatty acid transport" ...
  .. .. ..$ : chr [1:2] "rawp" "BH"
  ..$ kegg: num [1:5, 1:2] NA NA 0.8255 0.7448 0.0431 ...
  .. ..- attr(*, "dimnames")=List of 2
  .. .. ..$ : chr [1:5] "00250.Alanine, aspartate and glutamate metabolism" "00061.Fatty acid biosynthesis" "04620.Toll-like receptor signaling pathway" "04630.Jak-STAT signaling pathway" ...
  .. .. ..$ : chr [1:2] "rawp" "BH"
\end{Soutput}
\begin{Sinput}
> str(gseABS.int.holm)
\end{Sinput}
\begin{Soutput}
List of 1
 $ integrated:List of 2
  ..$ go  : num [1:5, 1:2] NA NA NA 0.598 NA ...
  .. ..- attr(*, "dimnames")=List of 2
  .. .. ..$ : chr [1:5] "GO:0004915.interleukin-6 receptor activity" "GO:0051902.negative regulation of mitochondrial depolarization" "GO:0005732.small nucleolar ribonucleoprotein complex" "GO:2000191.regulation of fatty acid transport" ...
  .. .. ..$ : chr [1:2] "rawp" "Holm"
  ..$ kegg: num [1:5, 1:2] NA NA 0.8255 0.7448 0.0431 ...
  .. ..- attr(*, "dimnames")=List of 2
  .. .. ..$ : chr [1:5] "00250.Alanine, aspartate and glutamate metabolism" "00061.Fatty acid biosynthesis" "04620.Toll-like receptor signaling pathway" "04630.Jak-STAT signaling pathway" ...
  .. .. ..$ : chr [1:2] "rawp" "Holm"
\end{Soutput}
\end{Schunk}

%\section{Discussion}

%\section{Conclusion}

\section{System Information}
Session information:

\begin{Schunk}
\begin{Sinput}
> sessionInfo()
\end{Sinput}
\begin{Soutput}
R Under development (unstable) (2011-08-10 r56704)
Platform: x86_64-apple-darwin10.8.0/x86_64 (64-bit)

locale:
[1] en_US.UTF-8/en_US.UTF-8/en_US.UTF-8/C/en_US.UTF-8/en_US.UTF-8

attached base packages:
[1] stats     graphics  grDevices utils     datasets  methods  
[7] base     

other attached packages:
[1] limma_3.9.16          GO.db_2.5.0           KEGG.db_2.5.0        
[4] org.Hs.eg.db_2.5.0    RSQLite_0.9-4         DBI_0.2-5            
[7] AnnotationDbi_1.15.16 RTopper_0.99.2        Biobase_2.13.9       

loaded via a namespace (and not attached):
[1] IRanges_1.11.26 MASS_7.3-14     multtest_2.9.1  splines_2.14.0 
[5] survival_2.36-9 tools_2.14.0   
\end{Soutput}
\end{Schunk}

\pagebreak
\section{References}
   \bibliographystyle{plain}
   \bibliography{./RTopper.bib}

\end{document}


